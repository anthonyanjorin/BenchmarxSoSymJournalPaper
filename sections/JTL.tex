\subsection{JTL}
\label{sec:JTL}

JTL (Janus Transformation Language)~\cite{CDEP10,ErPT18} is a model transformation language specifically tailored to support bidirectionality and non-determinism.
JTL is EMF compatible and provides tool support via an Eclipse plugin.

A transformation developer using JTL provides a set of constraints specifying a consistency relation.
Together with the metamodels, these constraints are transformed to an Answer Set Programming (ASP)~\cite{GL88} problem, which an ASP solver can use to enable consistency restoration.

JTL reuses QVT-R syntax \cite{QVT-1.3}, but deviates from its semantics deliberately and significantly. In particular, JTL maintains a persistent trace, in contrast to the purely state-based design of QVT-R. Second, JTL takes non-determinism into account by generating all possible models which may be obtained by selecting one of multiple rules being applicable to a given match. From these candidates, the user has to select the desired result, which is continued to be used in further synchronizations.


\subsubsection{Classification}
A summary of the features of JTL is provided in table~\ref{tab:features-all-tools}.
JTL is \emph{restoration-based}: the underlying constraint solver takes an inconsistent state of all models and figures out how to restore consistency.
The exact delta applied is irrelevant for this strategy.
%
With respect to horizontal and vertical inputs, therefore, JTL realizes an \emph{initial-diag-based} architecture, handling correspondences as additional constraints. 
%
JTL guarantees correctness with respect to the specified set of constraints, and hippocraticness as the current state of all models can always be taken as a solution if it already fulfills all constraints (and is thus already consistent).
%
JTL requires an \emph{explicit} consistency relation defined as a set of constraints (Prolog-like ``rules'').
%
Consistency restoration is \emph{implicit}:  the underlying constraint solver automatically derives \emph{fCR} and \emph{bCR} based on the provided constraints.
%
JTL currently supports \emph{directed}, \emph{on-demand}, and \emph{interactive} synchronization (which reduces to automatic synchronization in the case of deterministic transformations).

\lstdefinelanguage{jtl}{
	morekeywords = {transformation,top,relation,enforce,domain,where,when},
	morecomment=[l]{//},
	morecomment=[s]{/*}{*/}
}
\begin{lstlisting}[label={lst:jtl}, float=b!, language=jtl, escapechar=|, caption={The Families-to-Persons in JTL}]
transformation Families2Persons(
  family: Families, person: Persons) {|\label{lst:transf}|
  top relation FamilyRegister2PersonRegister {|\label{lst:FR2PR}|
    enforce domain family 
      fr : Families::FamilyRegister { };
    enforce domain person 
      pr : Persons::PersonRegister { };
    where {
      Father2Male(fr,pr);
      Mother2Female(fr,pr);
      Son2Male(fr,pr);
      Daughter2Female(fr,pr); }
  }|\label{lst:FR2PR_end}|	
  relation Father2Male {|\label{lst:F2M}|
    n: String;
    sn: String;
    enforce domain family 
      fr : Families::FamilyRegister {
        families = f : Families::Family {
          name = sn,
          father = m : Families::FamilyMember { 
            name = n 
          }
        }
      };
    enforce domain person 
      pr : Persons::PersonRegister {
        persons = m : Persons::Male {
          name = n + sn
        }
      };
  }|\label{lst:F2M_end}|	
  relation Mother2Female { ... };|\label{lst:M2F_end}| 	
  relation Son2Male { ... } |\label{lst:S2M}| 
  relation Daughter2Female { ... } |\label{lst:D2F}| 
}
\end{lstlisting}

\subsubsection{Benchmark solution with JTL}

The Families-to-Persons case implemented in JTL is depicted in listing~\ref{lst:jtl}.
JTL adopts a QVT-R~\cite{QVT-1.3} like textual concrete syntax.
On line~\ref{lst:transf}, variables \code{family} and
\code{person} are declared to match models conforming to the Families and Persons metamodels, respectively.
When the top relation \code{Family\-Register\-2\-Person\-Register} holds, then the two models are consistent.
In the \code{where} block, further relations can be specified as required conditions for the top relation to hold.
In this case, the top relation holds when the relations \code{Father\-2\-Male}, \code{Mother\-2\-Male}, \code{Son\-2\-Male}, and \code{Daughter\-2\-Male}, all hold.
Variables bound in one relation, such as \code{fr} and \code{pr}, can be passed to invoked relations.
In addition to structural constraints represented as object patterns in the relations, it is also possible to specify attribute constraints, e.g., on line 29.

Per default, JTL handles non-determinism by generating all possible models and allowing the user to iterate through them one by one.
To implement the benchmark, which requires runtime configuration to make the tests deterministic, additional constraints have to be provided so the solver can determine the desired -- and in the case of the benchmark -- unique solution.
Currently, these additional constraints have to be specified on a comparably low level of abstraction as direct input for the underlying constraint solver, and not with the normal JTL concrete syntax.



%%% Local Variables:
%%% mode: latex
%%% TeX-master: "../main"
%%% End:
