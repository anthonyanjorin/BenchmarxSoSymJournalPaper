\section{Related Work}
\label{sec:RelatedWork}

In this section, we provide a discussion of related work divided into two broad groups:  (i)~existing results concerned with providing benchmark frameworks and examples in an MDE context, and (ii)~previous work on classifying and comparing bx approaches.
The restriction of (i) and (ii) to MDE and bx, respectively, is to keep the scope of our discussion manageable.

\subsection{Benchmark frameworks and examples}
\label{sec:BenchmarkFrameworks}

% Benchmarx
To the best of our knowledge, Benchmarx is the first and only framework for developing and executing benchmarks for bidirectional transformations.
The framework is based on initial conceptual work regarding the requirements bx benchmarks and bx benchmark frameworks should satisfy~\cite{AnjorinCG0RS14}.
An implementation of these concepts was developed several years later and described in a paper for the BX 2017 workshop~\cite{Anjorin2017}.
This article goes considerably beyond the preliminary work on Benchmarx as it presents the selected benchmark case and its challenges more accurately, includes a significantly extended section on the underlying conceptual framework regarding bx tool architectures, and presents and compares a broad spectrum of solutions to the selected case, demonstrating the applicability of the Benchmarx framework to heterogeneous bx tools. 

% TTC
The Transformation Tool Contest (TTC)\footnote{http://www.transformation-tool-contest.eu} series have been organized to promote the comparison and evaluation of model tranformation tools.
Over the years the TTC has established conventions and guidelines for case descriptions and a systematic comparison of submitted solutions.
While there have been bx case submissions to the TTC before -- including our TTC case for the Families-to-Persons benchmark~\cite{Anjorin2017a} -- the TTC does not focus exclusively on bx and thus provides neither specialized infrastructure nor extra support for bx as the Benchmarx framework does.
The TTC is also a \emph{contest} typically with a ranking of solutions to identify winners and losers.
Due to the heterogeneity of bx approaches and tools, our goal in this paper is more to understand fundamental differences and similarities without claiming which solution is ``best''.

% SHARE
The SHARE environment~\cite{DBLP:journals/procedia/GorpM11} (Sharing Hosted Autonomous Research Environments) provides general support for sharing research tools via virtual machines. SHARE has been used as a support environment in the TTC series to make benchmark solutions accessible without imposing any installation effort for inspecting and executing solutions.
Solutions based on the Benchmarx framework may be distributed via SHARE or other virtual environments.
Benchmarx and SHARE thus satisfy orthogonal needs.   

% GT benchmarks
There have been numerous proposals for benchmarking MDE technology.
Varró et al.~\cite{DBLP:conf/vl/VarroSV05} suggest a suite of examples for graph transformation tools, chosen carefully to test various features related to graph pattern matching and rule-based model transformation.
While there are bx approaches based on graph transformation, this benchmark cannot be directly applied to benchmarking bx solutions.

% Incr benchmarks
Bergmann et al. extend the graph transformation benchmark of Varró et al. by examples specifically focussed on \emph{incremental} graph pattern matching~\cite{DBLP:conf/gg/BergmannHRV08}.
Although one of the examples in this extension is a bx problem, the benchmark itself covers features specific to incremental pattern matching and cannot be applied to benchmarking the broad spectrum of bx solutions.

% BigMDE
With a special focus on evaluating and promoting the \emph{scalability} of MDE tools, there have been several benchmark proposals from the BigMDE workshop series: Strüber et al. present a collection of examples and a conceptual framework for evaluating solutions~\cite{DBLP:conf/staf/0001KAPR16}.
Strüber et al. argue to evaluate not only the scalability of transformations (performance) but also the scalability of specifications (maintainability).
While the examples and features are not specific to bx, we have included the ``size'' of specifications as a factor for our comparison of bx solutions in section~\ref{sec:Evaluation}.
Additional, better metrics for measuring the complexity of bx specifications can and should be explored in the future.

Also as part of the BigMDE workshop, Izsó et al. present MONDO-SAM~\cite{DBLP:conf/staf/IzsoSRV14} as a framework to systematically assess MDE scalabilty.
MONDO-SAM takes a very broad view on benchmarking MDE technology, covering numerous MDE tasks.
While MONDO-SAM does not specifically cover bx benchmarking and its unique challenges, the Benchmarx framework could be aligned with MONDO-SAM in the future.

% Repos of bx examples
In recognition of the need to collect bx examples, the Bx Example Repository~\cite{Cheney2014} was set up and is continuously being extended and maintained~\footnote{\url{http://bx-community.wikidot.com/examples:home}}.
The repository includes a short description of the Families-to-Persons case, which we selected as an initial example for demonstrating the feasibility of the Benchmarx framework.
This case, of which several variants exist, was originally proposed as part of the ATL~\cite{SCP-Jouault2008} transformation zoo\footnote{\url{http://www.eclipse.org/atl/atlTransformations/\#Families2Persons}}.
For its implementation with the Benchmarx framework, the case was refined into a bidirectional incremental transformation with a configurable backward transformation, for which we developed a comprehensive set of test cases.
The Benchmarx framework is meant to complement the Bx Example Repository.
With time, the most promising examples in the repository can be chosen and suitably extended to establish them with the Benchmarx framework as bx benchmarks.

% Collection of examples
In order to enable a more comprehensive evaluation, bx tools should be compared with the help of a \emph{spectrum} of benchmarks rather than with the help of just a single case (currently Families-to-Persons).
In fact, more benchmarks are already available for evaluation with the Benchmarx framework, but have been implemented only in a few bx tools to date.
Procuring further solutions is currently ongoing work.
For example, all cases proposed by Westfechtel~\cite{SoSym2018-Westfechtel} have been implemented as bx benchmarks in the Benchmarx framework.
While these cases were originally designed for evaluating the bx language QVT Relations (QVT-R ~\cite{QVT-1.3}), they can be meaningfully applied to other bx languages and tools.


\subsection{Classification and comparison of bx approaches}
\label{sec:ClassificationOfBxApproaches}

% Zinovys work - tiles as formal schematic classification
There has been a considerable amount of research done on classifying and comparing bx approaches.
As an example of work towards a formal categorization of bx, our collection of bx tool architectures in section~\ref{sec:Foundations} is inspired by the formal \emph{tile} framework of Diskin; the interested reader is referred to his seminal work~\cite{DBLP:conf/gttse/Diskin09} for a more rigourous handling of synchronization operations and their composition.

% Bidirectional languages
% TGG comparison papers
Besides proposed formal frameworks for bx, there has also been work on comparing \emph{similar} bx approaches:  Foster et al. discuss and compare different and complementary approaches to bidirectional programming~\cite{DBLP:conf/ssgip/FosterMV10}.
With similar goals, there have also been papers comparing numerous TGG tools~\cite{DBLP:journals/eceasst/HildebrandtLGRGSLAS13,DBLP:journals/eceasst/LeblebiciASHRG14}.
In both cases, however, the comparison is restricted to relatively homogenous variants of the same general bx approach.
This allows for a detailed comparison but is orthogonal to our goal of comparing \emph{diverse} bx approaches with the Benchmarx framework.

% JTL vs. TGG
There have also been some attempts to compare rather different bx approaches, such as the comparison of JTL and some TGG tools provided by Eramo et al.~\cite{DBLP:journals/eceasst/EramoB13}.
This paper can be seen as a consequent development in the same direction of such ad-hoc comparisons, but with the crucial difference that we now provide both a conceptual and technical framework for establishing such bx frameworks in the future.  

% Romina - Taxonomy (kind of glossary of definitions and requirements)
The Bx Community\footnote{http://bx-community.wikidot.com} has also been working towards establishing standard terminology and a classification schema for bx.
Eramo et al. take first steps towards a taxonomy for bx~\cite{DBLP:conf/sattose/EramoMP14} by providing a collection of basic definitions and requirements for bx approaches.

% Soichiro - detailed feature model for bx
With similar goals, Hidaka et al. provide a comprehensive feature-based classification~\cite{SOSYM-Hidaka2016} of bx approaches.
Compared to the feature model proposed by Hidaka et al., our feature model focuses on a relatively small set of ``core features'' and does not intend to provide a comprehensive classification of the bx landscape.
Our feature model is not just a subset of Hidaka's feature model, but introduces different features and organizes them in a different way.
Essentially, as far as the bx tool architecture is concerned (see the left-hand side of figure~\ref{fig:featureModelBxTools}), our feature model precisely reflects the conceptual framework introduced in section~\ref{sec:Foundations} and constitutes an original contribution; the remaining parts are covered by Hidaka's feature model but in much less detail due to the large number of features covered.

Other results towards classifying bx tasks and scenarios include work by Diskin et al.~\cite{DBLP:journals/jss/DiskinGWC16} on a taxonomy for bidirectional model synchronization, and Lämmel's megamodelling approach to characterizing different bx scenarios~\cite{DBLP:conf/sle/Lammel16}.
These results are complementary to our Benchmarx framework as their goal is to provide a \emph{tool-independent} classification of bx problems, while we focus in this paper on classifying diverse bx \emph{solution} strategies, tools, and approaches.   

%%% Local Variables:
%%% mode: latex
%%% TeX-master: "../main"
%%% End:
