\subsection{BXtend}
\label{sec:BXtend}

\NOTE{\emph{Solution expert:} Thomas, \emph{Interviewer:} Tony}

BXtend~\cite{MODELSWARD2018-Buchmann}, an acronym for \emph{Bidirectional Xtend}, is a pragmatic approach to programming bx, developed to address problems encountered in the practical application of existing bx languages and tools~\cite{DBLP:conf/icsoft/BuchmannG16}.

BXtend provides an \emph{internal DSL} based on Xtend, a multi-paradigm language supporting object-oriented, procedural and functional programming, as well as a seamless integration with the Eclipse IDE and the mainstream programming language Java.

To realize a bidirectional transformation, the transformation developer defines a \emph{correspondence model}, resembling the correspondence graph of a \emph{triple graph grammar} \cite{Schurr1994}.
In contrast to triple graph grammars, however, transformation rules are defined in a procedural rather than declarative way and are attached to correspondences.
Both transformation directions are programmed separately; the transformation developer is responsible for programming both directions in a mutually consistent way.
Altogether, BXtend provides a light-weight framework in which developers can structure their bx solutions in a straightforward manner.

\lstdefinelanguage{bxtend}{
	morekeywords = {class, override, extends, new, null, as, typeof, val, if, else, def},
	morecomment=[l]{//},
	morecomment=[s]{/*}{*/}
}

\begin{lstlisting}[label={lst:bxtendSoln}, float=htb!, language=bxtend, caption={Rule orchestration and a specific rule with BXtend}]
class Family2PersonTransformation {
  ...
  def private void addRules() {
    ...
    rules.add(new Register2Register(
      sourceModel, targetModel, 
      corrModel, decision))
    rules.add(new MotherDaughter2Female(
      sourceModel, targetModel, 
      corrModel, decision))
    rules.add(new FatherSon2Male(
      sourceModel, targetModel, 
      corrModel, decision))
  }
  ...
}

class MotherDaughter2Female 
  extends FamilyMember2Person {
  ...
  override sourceToTarget(String s) {  
    sourceModel.allContents
      .filter(typeof(Family))
      .forEach [ family |
        val corr = family.eContainer
                         .getCorrModelElem
        val females = ECollections.newBasicEList
        females.addAll(family.daughters)
        if (family.mother != null) 
          females.add(family.mother)
        females.forEach [ member |
          val corrFemale = 
            member.getOrCreateCorrModelElement(s)
          val female = getOrCreatePersonElement(
	        family.name + ", " + member.name, 
	        corrFemale, ...)
          (corr.personElement as PersonRegister)
            .getPersons().add(female)
        ]
      ]
  }

  override targetToSource(String s) {
    targetModel.allContents
               .filter(typeof(Female))
               .forEach [ p | ...]
  }
}
\end{lstlisting}




\subsubsection{Classification}

BXtend's architectural style is \emph{restoration-based}, addressing \emph{initial-diag-based} application scenarios (see table~\ref{tab:features-all-tools}).
The re\-storation-based architecture for initial-diag-based application scenarios is depicted to the left of figure~\ref{fig:initialDiagBased}:  the programmer takes a diag and has the task of manipulating the dependent model until both models are consistent again.
The corr between the models should also be updated in the process and returned.

With BXtend, the program representing \emph{fCR/bCR} is decomposed into multiple ``rules'' consisting of restoration logic separated into forward and backward direction.
Each rule defines flexibly, e.g., based on a certain type, if it is applicable and can be used to check and fix a specific inconsistency or not.
Similarly, each rule checks first by exploiting the supplied diag if existing structure in the dependent model can be reused, suitably changed, or must be deleted and created as required.
To perform \emph{fCR/bCR} on the top-most level, an orchestration component is implemented to decide the order in which individual rules are applied and combined to realize \emph{fCR/bCR}.
This global component can also perform a final clean up if required.
The delta implicitly induced as output (see figure~\ref{fig:initialDiagBased}) is mixed in the sense that each rule can freely perform deletion and creation as required in no fixed order.

BXtend provides no formal guarantees: both synchronization directions are implemented separately and are free to contradict each other.
This can be viewed positively: the bx programmer is free to do whatever is required to solve the current task.
%
There is no explicitly specified notion of consistency, i.e., this is \emph{implicitly} given by the implemented pair of \emph{fCR/bCR}.
%
The transformation developer has full \emph{explicit} control over consistency restoration, i.e., implements both \emph{fCR} and \emph{bCR} explicitly in Xtend.
%
BXtend is currently designed to support \emph{directed} synchronization, performed \emph{on-demand}, in an \emph{automatic} manner.

\subsubsection{Benchmark solution with BXtend}

To provide an impression of the benchmark solution with BXtend, listing~\ref{lst:bxtendSoln} depicts two classes:  \texttt{Family\-2\-Person\-Transforma\-tion} representing the orchestrating component, and \texttt{Mother\-Daughter\-2\-Female} representing a BXtend rule.
For the Families-to-Persons benchmark, the same ordering of rules can be used in both directions (lines 5--13).
This order is specified in \texttt{addRules()}; three rules are created and added following the composition hierarchy of the metamodels (containers first).
While the bx developer is free to program arbitrarily complex rule orchestration, current experience indicates that such a fixed order along the composition hierarchy is sufficient in most cases.


The specific rule \texttt{Mother\-Daughter\-2\-Female} checks for and handles, as its name suggests, inconsistencies between mothers/daughters in the source model and females in the target model.
The two methods in this class \texttt{source\-To\-Target} and \texttt{target\-To\-Source} indicate that both synchronization directions (\emph{fCR} and \emph{bCR}) are explicitly specified.
The first step in \texttt{source\-To\-Target} is to filter for all families, and to accumulate all ``female'' family members, i.e., a mother if present, and all daughters.
On line 32--33, the provided diag is accessed and used to retrieve the person connected to each female family member.
With a helper method \texttt{get\-Or\-Create\-Person\-Element}, this person is either created if necessary, or updated as required.

%%% Local Variables:
%%% mode: latex
%%% TeX-master: "../main"
%%% End:
