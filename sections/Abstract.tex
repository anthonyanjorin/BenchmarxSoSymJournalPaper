\begin{abstract}
	\NOTE{\emph{Responsible: } Bernhard}
	
	Bidirectional transformations (bx) are required in a wide range of application domains. While bx problems may be solved with unidirectional languages and tools, separate realization of forward and backward transformations with mutually consistent behavior proves difficult, laborious, and error-prone. In response to these problems, dedicated languages and tools for bx have been developed. However, due to their heterogeneity, approaches to bx are hard to compare. This motivates the need for benchmarks which facilitate the comparison of bx approaches.  
	
	This paper presents Benchmarx, a novel framework for evaluating bx tools. In particular, the framework takes the heterogeneity of bx tools into account by abstracting from technological spaces, tool architectures, and the data maintained by the tools. A benchmark for a specific bx problem is implemented by providing a suite of test cases. A solution to the bx problem may then be developed in a specific bx tool. By implementing a set of interfaces, the benchmark may be executed in the bx tool. In this way, the bx tool may be evaluated with respect to conciseness (size of the transformation definition), correctness (passed vs.\ failed test cases), and performance (runtime).
	
	To illustrate the use of the Benchmarx framework, we present a variety of solutions to the well known Families to Persons benchmark, which demands for synchronizing a database of families consisting of mother, father, daughters, and sons with a database containing a flat set of male and female persons. These solutions were selected to demonstrate and compare considerably different approaches to solving bx problems, ranging from procedural approaches with pairs of unidirectional transformations at one end to declarative bidirectional approaches at the opposite end of the spectrum. 
	
	\keywords{Bidirectional transformation \and Benchmark}
\end{abstract}
