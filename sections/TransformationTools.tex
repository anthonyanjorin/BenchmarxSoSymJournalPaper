\section{Transformation Tools}
\label{sec:TransformationTools}

\NOTE{\emph{Length:} 2 p., \emph{Responsible:} Thomas}

\NOTE{Classification according to previously introduced taxonomy. One short paragraph per tool.}

\NOTE{Tentative paragraph for EVL+STrace}

\emph{EVL+STrace} \cite{IST2018-Samimi} is based on EMF and the Epsilon framework \cite{epsilon}. A transformation definition is composed of a trace metamodel, which depends on the source and target metamodels, a set of query and update operations defined in EOL, the Epsilon Object Language, and a set of constraints written in EVL, the Epsilon Validation Language. Each constraint is directed, checks for an inconsistency between one of the participating models and the trace model, and repairs this inconsistency by updating both the trace model and the opposite model. Thus, the consistency relation between source and target models is defined by unidirectional checks, and consistency restoration is defined procedurally by the fix parts of the constraints. EVL+STrace supports synchronization between source and target models on demand, making use of a persistently stored trace model. Neither s- nor o-deltas are required. Synchronization is performed interactively by propagating changes in both directions. Automatic synchronization is possible, but requires (mechanical) rewriting of EVL constraints. 

\NOTE{Tentative paragraph for NMF}

\emph{NMF Synchronizations} \cite{SoSyM2017-Hinkel} is a bx language which has been realized as internal domain-specific language (DSL) in C\#. To specify a synchronization between a source and a target model, the transformation developer has to define a symmetric consistency relation between these models in terms of coupled consistency relations between their elements. The consistency relation is specified declaratively with the help of functions being free of side-effects. In addition, the transformation developer has to define consistency restoration with the help of procedures updating the model states in cases where there is no default restoration available. A transformation defined in NMF Synchronizations may be executed in different modes and direction. The transformation engine performs live synchronization; therefore, it requires o-deltas as input. A correspondence model is constructed at runtime, and may be accessed in the transformation definition; however, it is not stored permanently.